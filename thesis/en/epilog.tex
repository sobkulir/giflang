\chapter*{Conclusion}
\addcontentsline{toc}{chapter}{Conclusion}
At the end of the thesis we revisit our goals set in Section \ref{chap1:thesis_goals} and discuss how successful we were at completing them.

\begin{enumerate}
\item \textit{Giflang language design} 
   \begin{enumerate}[label=(\alph*)]
     \item \textit{Choose a suitable language type (Object-Oriented, functional, \ldots)} \\
     -- Pupils at the programming camps where this language is primarily going to be used mostly program in Python. Based on that we chose to create an
     Object-Oriented language that closely resembles Python.
     \item \textit{Define syntax, semantics and basic standard library} \\
     -- In the thesis, we described the language concepts. The language manual with examples can be found in the implementation itself. The basic standard
     library has structures usually found in dynamic languages of this kind -- objects, arrays and strings. Their methods are similar to JavaScript,
     because we represent them using their JavaScript counterparts.
   \end{enumerate}
\item \textit{Client-side Interpreter}
   \begin{enumerate}[label=(\alph*)]
	 \item \textit{Design and implement an interpreter for Giflang} \\
     -- We managed to implement the interpreter in JavaScript with the aid of a parsing library called Jison \cite{Jison}. The interpreter
     has a straightforward structure -- it builds an AST and then interprets individual nodes using the Visitor pattern.
     \item \textit{Support code-stepping with information about the current position in the source code, the call stack and the environment} \\
     -- This goal turned out to be trickier than we first anticipated. We had to find a construct in JavaScript that would allow us to wait until a certain
     condition is met. In this case the condition being that user has clicked on the `Next step' button. We explored several ways of implementing this including
     building custom continuations, async/await, using JavaScript's standard Atomics API and traversing the AST iteratively. Regarding the compatibility and
     performance, the best option would be to iteratively traverse the AST. However, since we were adding the code-stepping into the interpreter only after we
     implemented it using the recursive AST traversal, we decided to use the Atomics API to support code-stepping.

	 \item \textit{Implement an API for a standard I/O} \\
     -- We had two options while implementing the standard I/O. We could either expect users to provide the whole input prior to the execution or allow users
     to interactively provide input. The interactive I/O requires stopping the execution until users provide an input. We opted for the interactive version as we
     could reuse the findings from building the code-stepping.
   \end{enumerate}
\item \textit{Web IDE}
   \begin{enumerate}[label=(\alph*)]
     \item \textit{Design and implement an editor with characters replaced by images and support basic editing operations such as Delete,
     Backspace and cursor movement using the arrow keys} \\
     -- We managed to create an editor as described in the goal. Additionally, we added line numbers component that helps with orientation in the code.
	 \item \textit{Allow specifying input to a running program and show its output using images} \\
     -- We implemented an Input/Output boxes where users can perform I/O operations. We reused components used for the editor to display text as
     images.
	 \item \textit{Support code-stepping with a visualization of the currently executed position in the source code} \\
     -- We were able to implement the highlighting of a currently executed node in the AST. However, environment values and call stack
     is displayed in the text form rather than in the image form. Additionally, object properties are not displayed. We did not consider adding this functionality
     technically challenging and we rather focused on the parts like code-stepping during the last development period.
	 \item \textit{Allow storing programs on the server and loading the existing ones} \\
     -- We integrated a proprietary solution called Firebase \cite{Firebase} (by Google) and store the programs in its database. This allowed us to quickly create
     a working prototype without the need for a custom database and server.
	 \item \textit{Allow changing images in the mapping of images to characters and keywords} \\
     -- Our intent was to allow users to upload their images and change them dynamically. However, we did not manage to implement this functionality
     in time. It is easy to swap out the images, but the website needs to be run locally in order to do so.
   \end{enumerate}
\end{enumerate}

\subsection*{Future Work}
\begin{itemize}
   \item \emph{Allow language forks}

      As already mentioned in the previous list, we would like to support user-created mappings of images to characters in the future. Users would be able
      to view Giflang sources in different mappings by specifying a mapping ID as a part of the URL.

   \item \emph{Debugger with breakpoints}

      The code-stepping only supports stepping the code from the beginning until the end of the program. In the future, users should be able to create breakpoints
      and be able to run the code until some breakpoint is hit.

   \item \emph{Improve the UI}

      The current UI is very minimal and contains almost no styling; for example it would be nice to support resizable editor areas, as currently the sizes
      are hard-coded.
   \item \emph{Create Challenges} 

      In order to motivate users to program in Giflang, programming challenges can be integrated into the website. The challenges can be gradually harder problems,
      while the platform also contains a \emph{judge} that tests the user's code against predefined inputs and outputs.
   \item \emph{Transpiler to JavaScript or Python}

      Since the execution of Giflang is currently limited to the browser IDE, a transpiler of Giflang to JavaScript or Python would allow users to code in Giflang
      and later run their programs in standard environments. 
\end{itemize}
