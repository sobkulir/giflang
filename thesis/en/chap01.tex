\chapter{Introduction}

Traditional programming languages are defined in terms of plaintext tokens. These tokens
are then used in grammar rules to specify which orderings of tokens are valid programs and
which are not.
<TODO: Example>

However, tokens do not have to consist of plaintext necessarily. Plaintext characters can be substituted for
images, videos, sounds, etc. In this thesis we construct a programming language with characters and some
of the tokens substituted for short looping videos or images (GIFs, WEBP). On an image below you can see a~sample
program in Giflang\footnote{We named the programming language developed in this thesis Giflang}.
<TODO: Example>
TODO: Explanation of what the sample code does.

In order to be able to interpret and store programs like this we also define a textual representation. We discuss
this in section /TODO/. E.g. the program from Fig. TODO can be stored as:

/TODO: Code/

You can see that there is a clear mapping between images and semicolon-delimited tokens. Therefore a byproduct of
this thesis is a token-level programming language, i.e. a language that consists of delimited tokens. This tokens
can have different presentational forms as mentioned before.

A language like this one is of course impractical and hard to follow. We do not intend to create a language
for practical use. Our $3$ main motivations for creating Giflang are:
\begin{enumerate}
\item An image-based programming language like this does not exist yet. We discuss other existing graphically oriented
languages in section /TODO/. 
\item Using it in bootcamps for elementary and high schoolers. We co-organize a long-term Computer Science competition
named Prask\footnote{prask.ksp.sk todo}. After each semester we invite best contestants to a week-long bootcamp
filled with creative games mostly related to Computer Science. Giflang can be used as part of the games.
\item Allowing users to substitute images for their own can give younger users feeling of creating ''their own language''.
\end{enumerate}

Since we use images instead of characters we need to implement our own IDE to be able to create and run Giflang programs. Typically,
IDEs are desktop applications. However, since ease of access and use have bigger priority than efficiency for this application
we decided to move everything to a browser. Web technologies have improved significantly in the last decade\footnote{E.g. V8, new Ecmascript standards, web workers, etc}
and also users 

There are numerous programming languages that present flow of a program in a different way
from traditional plaintext. For example <scratch>...
\cite{Andel07}

\section{Related work}

\section{IDE as desktop vs browser application}

\section{Thesis goals}
